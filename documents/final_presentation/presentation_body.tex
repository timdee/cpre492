\lecture[1]{Remotely Connected Electric Field\\ Generator}{lecture-text}

\subtitle{for Particle Separation in a Fluid}

\date{27 April 2016}


\begin{document}

\begin{frame}
  \maketitle
\end{frame}

\section{Background}

\begin{frame}{Dielectrophoresis (DEP)}
  \begin{enumerate}
  \item A dielectric particle in a non uniform 
  electric field experiences a force
  \item Different potential fields and frequencies 
  has an effect on the net force
  \item First studied in 1950s by Herbert Pohl
  \item Recently revived due to the ability to
  manipulate micro-particles and cells. 
  \end{enumerate}
\end{frame}

\begin{frame}{Real World Application}
  \begin{enumerate}
    Potential to separate particles in spinal fluid
    Act as filter
    Research in separating cancerous cells from healthy cells
    Platelets from whole blood
    Red and white blood cells
    Strains of bacteria and viruses
  \end{enumerate}
\end{frame}

\begin{frame}{Our Project}
  \begin{enumerate}
    A system to aid in the research of DEP
    Allow for quicker setup times
    Control Voltage and Frequency via the web
     1 to 60 VPP
    10k to 1Mhz
    Hold output for long time periods
    Brandon
    Small Form Factor
    Easy to use
    Plug and play
  \end{enumerate}
\end{frame}

\begin{frame}{Project Structure}
  \begin{enumerate}
    Raspberry Pi
    Web Interface
    Web Server
    Frequency Control Solution
    Voltage Control Solution
  \end{enumerate}
\end{frame}

\section{Initial Implementation}

\begin{frame}{Project Progression}
  \begin{enumerate}
    Initial Implementation
    Raspberry Pi 
    Host web server
    Remote manipulation of circuit output
    Web interface can provide additional functionality
    GPIO pins input to circuit
    Circuit Output
    Frequency generated by GPIO pin 
    GPIO waveform integrated to get sine wave
    Sine wave amplified to form output
  \end{enumerate}
\end{frame}

\section{Intermediate Implementation}

\begin{frame}{Project Progression}
  \begin{enumerate}
    Intermediate Implementation
    Raspberry Pi controls Integrated circuit components
    Minigen Signal Generator
    SPI communications
    Produces frequency 10 Khz - 4 Mhz
    Digital Potentiometers
    SPI communications
    Vary resistance to control amplifier
    Amplifier controls voltage output from circuit
  \end{enumerate}
\end{frame}

\begin{frame}{Problems and Setbacks}
  \begin{enumerate}
    Mosfet Amplifier
    Digital Potentiometer
    Resistance drops with AC signal
    Distorted the sine wave
    Op Amps
    Slew Rates
    Gain Bandwidth
    Minigen
    B23 Bug
  \end{enumerate}
\end{frame}

\begin{frame}{Digital Potentiometer Amplifier Circuit }
  \begin{enumerate}
    \item "image"
  \end{enumerate}
\end{frame}

\begin{frame}{MOSFET Amplifier}
  \begin{enumerate}
    \item "picture"
    \item information
  \end{enumerate}
\end{frame}

\begin{frame}{Problems and Setbacks}
  \begin{enumerate}
    Lost a group member
    BJT Switch
    Control through GPIO pin
    Current Leaks through when logically off
    Relay
    Operating Frequency not sufficient
    Brandon
    We have had to make quite a few adjustments from our original plan.
    This is especially the case with our digital potentiometers.
  \end{enumerate}
\end{frame}

\begin{frame}{SSR Circuit Implementation}
  \begin{enumerate}
    \item "image"
  \end{enumerate}
\end{frame}

\begin{frame}{Project Progression}
  \begin{enumerate}
    Current Design
    Raspberry Pi controls Integrated circuit components
    Minigen Signal Generator
    SPI communications
    Produces frequency 10 Khz - 4 Mhz
    Programmable Gain Amplifier (PGA)
    GPIO communications
    8 voltage options (0-7)
    Summing Amplifier
    Sums output from multiple PGA’s
    Three-Stages of PGA’s input to Summer
  \end{enumerate}
\end{frame}

\begin{frame}{Current Circuit Design}
  \begin{enumerate}
    \item "design diagram"
  \end{enumerate}
\end{frame}

\begin{frame}{Web Interface}
  \begin{enumerate}
    Hosted Locally
    Able to be seen on intranet
    Controls Voltage and Frequency
    Provides Additional Functionality
  \end{enumerate}
\end{frame}

\begin{frame}{Questions?}
  \begin{enumerate}
    Dielectrophoresis (DEP)
    Circuit Design
    Digital Potentiometer/ Operation Amplifier
    MOSFET/ Programmable Gain Amplifiers (PGA)
    Web Interface
    Design Documents
  \end{enumerate}
\end{frame}

\begin{frame}{Work Breakdown}
  \begin{enumerate}
    Initial Planning
    Project Website
    Reports
    Circuit Design
    Web Server
    SOC Communications
    PCB Design
  \end{enumerate}
\end{frame}

%\begin{frame}{Timeline}
%  \begin{enumerate}
%Completion Date
%Object
%Description
%10/1
%Project Plan
%Create a project plan which specifies the pieces of the project.
%10/15
%Design
%Complete a detailed design of each component of our project. Assign people to work on the %various pieces of the project.
%11/1
%Design Web Interface
%Design and build web interface. Outline code for Communications with Minigen and Digital %Potentiometers.
%11/15
%Hardware Communications and Design
%Get communications working between Raspberry Pi, Minigen, and Digital Potentiometers.
%12/1
%Amplifier Circuit
%Design and Testing
%Create voltage amplifier design and begin testing components of this design.
%12/7
%Presentation
%Present a prototype of our project.
%1/15 - 4/1
%Amplifier Circuit
%Implementation and Testing
%Attempt to complete voltage amplifier.
%4/15
%Documentation
%Create design document and user’s manual to describe how to utilize the functionality of the %project.    
%  \end{enumerate}
%\end{frame}
\end{document}