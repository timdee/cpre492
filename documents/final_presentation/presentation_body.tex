\lecture[1]{Remotely Connected Electric Field\\ Generator}{lecture-text}

\subtitle{for Particle Separation in a Fluid}

\date{27 April 2016}


\begin{document}

\begin{frame}
  \maketitle
\end{frame}

\section{Background}

\begin{frame}{Dielectrophoresis (DEP)}
  \begin{enumerate}
  \item A dielectric particle in a non uniform 
  electric field experiences a force
  \item Different potential fields and frequencies 
  has an effect on the net force
  \item First studied in 1950s by Herbert Pohl
  \item Recently revived due to the ability to
  manipulate micro-particles and cells. 
  \end{enumerate}
\end{frame}

\begin{frame}{Real World Application}
  \begin{enumerate}
    \item Potential to separate particles in spinal fluid
    \item Act as filter
    \item Research in separating cancerous cells from healthy cells
    \item Separate platelets from whole blood
    \item Separate red and white blood cells
    \item Strains of bacteria and viruses
  \end{enumerate}
\end{frame}

\section{Project Outline}

\begin{frame}{Project Description}
  \begin{enumerate}
    \item A system to aid in the research of DEP
    \item Allow for quicker setup times
    \item Control Voltage and Frequency via the web
      \begin{enumerate}
        \item 1 to 60 VPP
        \item 10k to 1Mhz
      \end{enumerate}
    \item Hold output for long time periods
    \item Small Form Factor
    \item Easy to use
    \item Plug and play
  \end{enumerate}
\end{frame}

\begin{frame}{Project Structure}
  \begin{enumerate}
    \item Raspberry Pi
    \item Web Interface
    \item Web Server
    \item Frequency Control Solution
    \item Voltage Control Solution
  \end{enumerate}
\end{frame}

\section{Initial Implementation}

\begin{frame}{Initial Implementation}
  \begin{enumerate}
    \item Raspberry Pi 
      \begin{enumerate}
        \item Host web server
        \item Remote manipulation of circuit output
        \item Web interface can provide additional functionality
        \item GPIO pins input to circuit
      \end{enumerate}
    \item Circuit Output
      \begin{enumerate}
        \item Frequency generated by GPIO pin 
        \item GPIO waveform integrated to get sine wave
        \item Sine wave amplified to form output
      \end{enumerate}
  \end{enumerate}
\end{frame}

\section{Intermediate Implementation}

\begin{frame}{Intermediate Implementation}
  \begin{enumerate}
    \item Raspberry Pi controls Integrated circuit components
    \item Minigen Signal Generator
    \item SPI communications
    \item Produces frequency 10 Khz - 4 Mhz
    \item Digital Potentiometers
    \item SPI communications
    \item Vary resistance to control amplifier
    \item Amplifier controls voltage output from circuit
  \end{enumerate}
\end{frame}

\begin{frame}{Problems and Setbacks}
  \begin{enumerate}
    \item Mosfet Amplifier
    \item Digital Potentiometer
    \item Resistance drops with AC signal
    \item Distorted the sine wave
    \item Op Amps
    \item Slew Rates
    \item Gain Bandwidth
    \item Minigen
    \item B23 Bug
  \end{enumerate}
\end{frame}

\begin{frame}{Digital Potentiometer Amplifier Circuit }
  \begin{enumerate}
    \item "image"
  \end{enumerate}
\end{frame}

\begin{frame}{MOSFET Amplifier}
  \begin{enumerate}
    \item "picture"
    \item information
  \end{enumerate}
\end{frame}

\begin{frame}{Problems and Setbacks}
  \begin{enumerate}
    \item Lost a group member
    \item BJT Switch
    \item Control through GPIO pin
    \item Current Leaks through when logically off
    \item Relay
    \item Operating Frequency not sufficient
    \item Brandon
    \item We have had to make quite a few adjustments from our original plan.
    \item This is especially the case with our digital potentiometers.
  \end{enumerate}
\end{frame}

\begin{frame}{SSR Circuit Implementation}
  \begin{enumerate}
    \item "image"
  \end{enumerate}
\end{frame}

%\begin{frame}{Project Progression}
%  \begin{enumerate}
%    \item Current Design
%    \item Raspberry Pi controls Integrated circuit components
%    \item Minigen Signal Generator
 %   \item SPI communications
 %   \item Produces frequency 10 Khz - 4 Mhz
 %   \item Programmable Gain Amplifier (PGA)
 %   \item GPIO communications
 %   \item 8 voltage options (0-7)
 %   \item Summing Amplifier
%    \item Sums output from multiple PGA’s
%    \item Three-Stages of PGA’s input to Summer
%  \end{enumerate}
%\end{frame}

\begin{frame}{Current Circuit Design}
  \begin{enumerate}
    \item "design diagram"
  \end{enumerate}
\end{frame}

\begin{frame}{Web Interface}
  \begin{enumerate}
    \item Hosted Locally
    \item Able to be seen on intranet
    \item Controls Voltage and Frequency
    \item Provides Additional Functionality
  \end{enumerate}
\end{frame}

\begin{frame}{Questions?}
  \begin{enumerate}
    \item Dielectrophoresis (DEP)
    \item Circuit Design
    \item Digital Potentiometer/ Operation Amplifier
    \item MOSFET/ Programmable Gain Amplifiers (PGA)
    \item Web Interface
    \item Design Documents
  \end{enumerate}
\end{frame}

\begin{frame}{Work Breakdown}
  \begin{enumerate}
    \item Initial Planning
    \item Project Website
    \item Reports
    \item Circuit Design
    \item Web Server
    \item SOC Communications
    \item PCB Design
  \end{enumerate}
\end{frame}

%\begin{frame}{Timeline}
%  \begin{enumerate}
%Completion Date
%Object
%Description
%10/1
%Project Plan
%Create a project plan which specifies the pieces of the project.
%10/15
%Design
%Complete a detailed design of each component of our project. Assign people to work on the %various pieces of the project.
%11/1
%Design Web Interface
%Design and build web interface. Outline code for Communications with Minigen and Digital %Potentiometers.
%11/15
%Hardware Communications and Design
%Get communications working between Raspberry Pi, Minigen, and Digital Potentiometers.
%12/1
%Amplifier Circuit
%Design and Testing
%Create voltage amplifier design and begin testing components of this design.
%12/7
%Presentation
%Present a prototype of our project.
%1/15 - 4/1
%Amplifier Circuit
%Implementation and Testing
%Attempt to complete voltage amplifier.
%4/15
%Documentation
%Create design document and user’s manual to describe how to utilize the functionality of the %project.    
%  \end{enumerate}
%\end{frame}
\end{document}