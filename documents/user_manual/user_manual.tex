%%%%%%%%%%%%%%%%%%%%%%%%%%%%%%%%%%%%%%%%%%%%%%%%%%%
%% LaTeX book template                           %%
%% Author:  Amber Jain (http://amberj.devio.us/) %%
%% License: ISC license                          %%
%%%%%%%%%%%%%%%%%%%%%%%%%%%%%%%%%%%%%%%%%%%%%%%%%%%

\documentclass[a4paper,11pt]{book}
\usepackage[T1]{fontenc}
\usepackage[utf8]{inputenc}
\usepackage{lmodern}
%%%%%%%%%%%%%%%%%%%%%%%%%%%%%%%%%%%%%%%%%%%%%%%%%%%%%%%%%
% Source: http://en.wikibooks.org/wiki/LaTeX/Hyperlinks %
%%%%%%%%%%%%%%%%%%%%%%%%%%%%%%%%%%%%%%%%%%%%%%%%%%%%%%%%%
\usepackage{hyperref}
\usepackage{graphicx}
\usepackage[english]{babel}

%%%%%%%%%%%%%%%%%%%%%%%%%%%%%%%%%%%%%%%%%%%%%%%%%%%%%%%%%%%%%%%%%%%%%%%%%%%%%%%%
% 'dedication' environment: To add a dedication paragraph at the start of book %
% Source: http://www.tug.org/pipermail/texhax/2010-June/015184.html            %
%%%%%%%%%%%%%%%%%%%%%%%%%%%%%%%%%%%%%%%%%%%%%%%%%%%%%%%%%%%%%%%%%%%%%%%%%%%%%%%%
\newenvironment{dedication}
{
   \cleardoublepage
   \thispagestyle{empty}
   \vspace*{\stretch{1}}
   \hfill\begin{minipage}[t]{0.66\textwidth}
   \raggedright
}
{
   \end{minipage}
   \vspace*{\stretch{3}}
   \clearpage
}

%%%%%%%%%%%%%%%%%%%%%%%%%%%%%%%%%%%%%%%%%%%%%%%%
% Chapter quote at the start of chapter        %
% Source: http://tex.stackexchange.com/a/53380 %
%%%%%%%%%%%%%%%%%%%%%%%%%%%%%%%%%%%%%%%%%%%%%%%%
\makeatletter
\renewcommand{\@chapapp}{}% Not necessary...
\newenvironment{chapquote}[2][2em]
  {\setlength{\@tempdima}{#1}%
   \def\chapquote@author{#2}%
   \parshape 1 \@tempdima \dimexpr\textwidth-2\@tempdima\relax%
   \itshape}
  {\par\normalfont\hfill--\ \chapquote@author\hspace*{\@tempdima}\par\bigskip}
\makeatother

%%%%%%%%%%%%%%%%%%%%%%%%%%%%%%%%%%%%%%%%%%%%%%%%%%%
% First page of book which contains 'stuff' like: %
%  - Book title, subtitle                         %
%  - Book author name                             %
%%%%%%%%%%%%%%%%%%%%%%%%%%%%%%%%%%%%%%%%%%%%%%%%%%%

% Book's title and subtitle
\title{\Huge \textbf{User Manual} \\ \huge A Setup Guide from Scratch}
% Author
\author{\textsc{Timothy Dee} \\ \textsc{Brandon McDonnel} \\ \textsc{Justin Long}} %\thanks{\url{www.example.com}}}


\begin{document}

\frontmatter
\maketitle

%%%%%%%%%%%%%%%%%%%%%%%%%%%%%%%%%%%%%%%%%%%%%%%%%%%%%%%%%%%%%%%
% Add a dedication paragraph to dedicate your book to someone %
%%%%%%%%%%%%%%%%%%%%%%%%%%%%%%%%%%%%%%%%%%%%%%%%%%%%%%%%%%%%%%%
%\begin{dedication}
%Dedicated to Calvin and Hobbes.
%\end{dedication}

%%%%%%%%%%%%%%%%%%%%%%%%%%%%%%%%%%%%%%%%%%%%%%%%%%%%%%%%%%%%%%%%%%%%%%%%
% Auto-generated table of contents, list of figures and list of tables %
%%%%%%%%%%%%%%%%%%%%%%%%%%%%%%%%%%%%%%%%%%%%%%%%%%%%%%%%%%%%%%%%%%%%%%%%
\tableofcontents
\listoffigures
\listoftables

\mainmatter

%%%%%%%%%%%
% Preface %
%%%%%%%%%%%
\chapter*{Preface}
\section*{Purpose}
The purpose of this manual is to allow a person to easily set up and begin using our system.
Topics covered include everything from installing an operating system on a Raspberry Pi to 
using our provided web interface to adjust voltage and frequency output of the device.
The design document for the project is also attached for better understanding of the functionality of
the system as a whole.

\section*{Structure of Manual}
% You might want to add short description about each chapter in this book.
This document lay's out the setup process from beginning to end.
Each step of the process is described in enough detail that
the process may be accomplished without the use of other reference materials.
No knowledge of linux systems, Raspberry Pi, or Apache Webserver is assumed.
At the end of this manual is attached the design document.

%\section*{About the companion website}
%The website\footnote{\url{https://github.com/amberj/latex-book-template}} for this file contains:
%\begin{itemize}
%  \item A link to (freely downlodable) latest version of this document.
%  \item Link to download LaTeX source for this document.
%  \item Miscellaneous material (e.g. suggested readings etc).
%\end{itemize}

%%%%%%%%%%%%%%%%%%%%%%%%%%%%%%%%%%%%
% Give credit where credit is due. %
% Say thanks!                      %
%%%%%%%%%%%%%%%%%%%%%%%%%%%%%%%%%%%%
%\section*{Acknowledgements}
%\begin{itemize}
%\item A special word of thanks goes to Professor Don Knuth\footnote{\url{http://www-cs-faculty.stanford.edu/~uno/}} (for \TeX{}) and Leslie Lamport\footnote{\url{http://www.lamport.org/}} (for \LaTeX{}).
%\item I'll also like to thank Gummi\footnote{\url{http://gummi.midnightcoding.org/}} developers and LaTeXila\footnote{\url{http://projects.gnome.org/latexila/}} development team for their awesome \LaTeX{} editors.
%\item I'm deeply indebted my parents, colleagues and friends for their support and encouragement.
%\end{itemize}
%\mbox{}\\
%\mbox{}\\
%\noindent Amber Jain \\
%\noindent \url{http://amberj.devio.us/}

%%%%%%%%%%%%%%%%
% NEW CHAPTER! %
%%%%%%%%%%%%%%%%
\chapter{Materials}

%\begin{chapquote}{Author's name, \textit{Source of this quote}}
%``This is a quote and I don't know who said this.''
%\end{chapquote}

\section{Raspberry Pi}
The Raspberry Pi is an inexpensive computing device.
The operating system for this device is installed on an SD card.
The Printed Circuit Board containing all of the hardware for this project will
be connected to the GPIO pins on this device.

\section{Printed Circuit Board}
Board containing our frequency and voltage regulating hardware.
The design document covers the hardware included on this PCB in great detail.
An overview of the functionality will be given here as a summary.
All that must be understood for operation of the system is the inputs and ouputs of this hardware component.
The imputs are the GPIO pins on the provided Raspberry Pi.
The outpus are wires which may be soldered on to the PCB at the locations described.
Circuit diagrams of the PCB are included in the attached design document.

\section{Software}
There are a few components to our provided software.
Included among these is a startup script which can be
run on a fresh install of Rasbian Linux operating system.
This startup script will create all necessary files
and make all necessary changes to the device.
After the startup script is run, the system should be fully operational
and the user should be able to control it from a web browser.

In the case that combatibility issues between the script and future versions of
Rasbian Linux operating system arrise in the future,
a fully detailed guide is provided to enable manual setup.
We will describe the necessary software packages used in this setup,
how to acquire these packages, and how to configure them similar to the way we have done.

\chapter{Setup and Configuration}
\section{Acquiring a Raspberry Pi}
The Raspberry Pi is a small computing device which may be purchased online for less than \$50.
Other items which may need to be purchased along with the Raspberry Pi include:

1. micro usb power cable

2. protective case

3. TODO

\section{Installing Rasbian}
Rasbian is a distribution of linux which has very light weight system requirements.
The operating system is optimized to run on the raspberry pi, and 
contains many useful packages reinstalled.

\section{Networking the Raspberry Pi}
In order for the Raspberry Pi to be controlable from the provided web interface,
the device must be connected to the same network as the controlling computer.

\section{Configuring the Raspberry Pi - Configure Script}
NOT YET IMPLEMENTED

Simply run the provided script as root user.
Place the folder containing cgi-bin, www, and setup\_script.bash whereever
you want the webserver to be hosted on the raspberry pi.
After this, enter the following command \it{./setup\_script.bash}.

\section{Configuring the Raspberry Pi - Manual Configuration}
\subsection{Install Packages}


\chapter{Using the Device}
\section{Physical Connections}
\section{Using the Web Interface}

\chapter{Troubleshooting}

\chapter{Design Document}

%%%%%%%%%%%%%%%%%%%%%%%%%%%%%%%%%%%%%%%%%%%%%%%%%%%%%%%
% Sample table                                        %
% Source: www1.maths.leeds.ac.uk/latex/TableHelp1.pdf %
%%%%%%%%%%%%%%%%%%%%%%%%%%%%%%%%%%%%%%%%%%%%%%%%%%%%%%%
\begin{table}[ht]
\caption{Sample table} % title of Table
\centering % used for centering table
\begin{tabular}{c c c c}
% centered columns (4 columns)
\hline\hline %inserts double horizontal lines
S. No. & Column\#1 & Column\#2 & Column\#3 \\ [0.5ex]
% inserts table
%heading
\hline % inserts single horizontal line
1 & 50 & 837 & 970 \\
2 & 47 & 877 & 230 \\
3 & 31 & 25 & 415 \\
4 & 35 & 144 & 2356 \\
5 & 45 & 300 & 556 \\ [1ex] % [1ex] adds vertical space
\hline %inserts single line
\end{tabular}
\label{table:nonlin} % is used to refer this table in the text
\end{table}

% Sample list
Duis aute irure dolor in reprehenderit in voluptate velit esse cillum dolore eu fugiat nulla pariatur. Excepteur sint occaecat cupidatat non proident, sunt in culpa qui officia deserunt mollit anim id est laborum. \\ Lorem ipsum list:
\begin{itemize}
\item Mauris sit amet nulla mi, vitae rutrum ante.
\item Maecenas quis nulla risus, vel tincidunt ligula.
\item Nullam ac enim neque, non \emph{dapibus} mauris.
\end{itemize}

\end{document}
